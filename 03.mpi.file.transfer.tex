\documentclass{article}
\title{FTP using MPI}
\author{Group x}
\begin{document}
\maketitle
\section{MPI Introduction}
MPI stands for Message Passing Interface - is a standardized and portable message-passing standard designed.
Typically, it allow developer design a parallel system which can transfer data to each other.

It sound similar to "fork" or "pthread" technique, but only when running on single machine. But the story is new when the parallel processes work in the distributed system like a cluster where the multiple of machines work together.
In this case, the mpi-driven program can be designed to spread parallel processes onto multiple machines easily and allow those parallel processes share data to each other.

For that reason, MPI used a lot to design parallel-driven program on distributed system.

\section{MPI and File Transfer}
A process to transfer begin when 1 side read amount of byte from source (we consider it is 1M) then send these data to destination.file.
There are 2 task cause slow down is when data read from disk and writen to disk. Normally, there processes work serialize. But it
can be optimized by convert this serialize process to parallel. And the key in here is a buffer. Basically, temporary storing
data to buffer before that data can be stored in dest is a step in all file transfer program. Then lets imagine a bit, while the disk try to fetch out data to fill up buffer before this buffer can be sent to another disk to save (we consider this is a distributed system has multiple disks). It will like 2 worker, worker A fetch data and fill up buffer, worker B get data from
buffer then store it to disk. During worker A do his job, worker B wait and vice versa. That cause delay, both 2 worker should work
concurrently. But they cannot because there are only 1 buffer, if A use buffer, B do not. And that why I say buffer is key point
to optimize this process. If we have more than 1 buffer then both 2 worker can work at same time.
And the parallel process make this come true, which mean 2 process parallel will have 2 buffer of their own. So, if 1 process do task read data, another can write data. That is idea, you improve job by giving more tool to worker. Serial program has only 1 
buffer used at once time but parallel program has 2. So parallel driven designed program is solution for optimizing file transfer 
task. But this is incomplete solution when data can not transfer between 2 buffers. If 2 processes run on 2 difference machines
then how can data jump from buffer of worker A to buffer of worker B. Simple idea, MPI design has already solve it. Actually, it
is made to solve this problem.
Conclusion, 2 main point to optimize file transfer process on distributed system have already exit in definition of MPI. Therefore,
Applying MPI in file transfer in distributed system is important.

\section{Methodology}

\end{document}