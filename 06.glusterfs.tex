\documentclass{article}
\usepackage{geometry}
\title{Prac 6: glusterfs - setup a trusted pool in linux}
\author{usth student}
\begin{document}
\maketitle
\section{Setup glusterfs in Archlinux repo}
First, the glusterfs package include both glusterfs server and gluster native client in arch repo,
so what you need to do is install it with pacman:
\begin{verbatim}
sudo pacman -S glusterfs
\end{verbatim}

Then the gluster daemon must be start on each peer/node machine
\begin{verbatim}
sudo systemctl start glusterd.service
\end{verbatim}

After running daemon, each machine become itself a trusted pool, the pool can be expanded by add
more node machine to pool by any inside pool node machine:
\begin{verbatim}
# define host name for ip address
sudo echo "127.0.0.1  gluster0" >> /etc/hosts
sudo echo "192.168.0.102  gluster1" >> /etc/hosts

# add new peer to trusted pool
sudo gluster peer probe gluster1
\end{verbatim}

Now, a new volume can be created from the trusted pool on any machine inside pool. For newest version of
glusterfs, the "stripe" function is deprecated, insteal, all new volume is set as distributed volume
by default. So the distributed-replicated type volume can be created by set number of replica node smaller
than actual number of bricks. Btw, the brick is simple a space in each node (considered as a directory).
\begin{verbatim}
sudo gluster volume create dr-volume replica 2 \
gluster0:/linux/gluster/brick1 \
gluster1:/home/lhud/Desktop/Share/glfs/b1 \
gluster0:/linux/gluster/brick2 \
gluster1:/home/lhud/Desktop/Share/glfs/b2
\end{verbatim}

The new volume created can be accessed after active
\begin{verbatim}
sudo gluster volume start dr-volume
\end{verbatim}

We can list out some information of volume
\begin{verbatim}
sudo gluster volume list
sudo gluster volume info dr-volume
\end{verbatim}

From any client, the volume can mount as a normal file system with type: glusterfs. the fuse module must
be actived before mounting:
\begin{verbatim}
sudo modprobe fuse
sudo mount -t glusterfs gluster0:/dr-volume /mnt
\end{verbatim}

This is some benchmarks measurement from copy file to volume
\begin{verbatim}
du -h linusThesis.pdf
460K	linusThesis.pdf

dd if=linusThesis.pdf of=/mnt/linusThesis.pdf
919+1 records in
919+1 records out
470861 bytes (471 kB, 460 KiB) copied, 0.136306 s, 3.5 MB/s
919+1 records in
919+1 records out
470861 bytes (471 kB, 460 KiB) copied, 0.0865043 s, 5.4 MB/s
919+1 records in
919+1 records out
470861 bytes (471 kB, 460 KiB) copied, 0.09253 s, 5.1 MB/s

du -h test.txt
6.3M	test.txt

dd if=test.txt of=/mnt/linusThesis.pdf
12875+1 records in
12875+1 records out
6592054 bytes (6.6 MB, 6.3 MiB) copied, 0.631172 s, 10.4 MB/s
12875+1 records in
12875+1 records out
6592054 bytes (6.6 MB, 6.3 MiB) copied, 0.615666 s, 10.7 MB/s
12875+1 records in
12875+1 records out
6592054 bytes (6.6 MB, 6.3 MiB) copied, 0.60536 s, 10.9 MB/s
\end{verbatim}
\end{document}
